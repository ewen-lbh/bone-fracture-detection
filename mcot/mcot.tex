% vim: set textwidth=120 :
\documentclass[a4paper]{article}
\usepackage[a4paper, total={6.5in, 9.5in}]{geometry}
\usepackage[utf8]{inputenc}
\usepackage[T1]{fontenc}
\usepackage[french]{babel}
\usepackage[bookmarks, hidelinks, unicode]{hyperref}
\usepackage[normalem]{ulem}
\usepackage{amsmath}
\usepackage{xfrac}
\usepackage{biblatex}
\usepackage{csquotes}
\usepackage{setspace}
\addbibresource{../sources.bib}
\title{Détection et classification de fractures osseuses}
\setlength{\parskip}{10pt}
\author{Ewen Le Bihan}
\date{2021-12-20}

\newcommand\wordcount[1]{
    \immediate\write18{texcount -sub=section \jobname.tex  | grep "Section" | sed -e 's/+.*//' | sed -n \thesection p > 'count.txt'}
    \begin{center}
        ($\sfrac{\input{count.txt}}{#1}$ mots)
    \end{center}
}

\begin{document}
\maketitle

\noindent
\begin{description}
    \item[Professeurs encadrants du candidat]  \hfill
        \begin{itemize}
            \item M. Chireux
            \item M. Lauront
        \end{itemize}
\end{description}

\setstretch{2}
\section{Ancrage au thème}

Les fractures osseuses constituent une catégorie de blessures assez communes mais potentiellement graves, avec des
conséquences parfois permanentes sur la suite de la vie du patient.  Plus particulièrement, la détection et
classification de fractures osseuses constitue une étape essentielle du traitement de tels blessures.

% Le sujet s'ancre donc dans le thème de la santé.

\wordcount{50}

\section{Motivation du choix}

Il convient de s'intérésser à l'automatisation du processus de détection et de classification de telles blessures, afin
de fluidifier l'orientation dans un service hospitalier. En effet, les hopitaux sont ces derniers temps souvent en sous-effectif, 
et fournir un premier diagnostic sans intervention professionnelle pourrait améliorer la situation. 

\wordcount{50}

\section{Positionnements thématiques}

\setstretch{1}
\begin{itemize}
    \item \emph{INFORMATIQUE (Traitement d'images)}
    \item \emph{INFORMATIQUE (Aprentissage automatique)}
    \item \emph{MATHÉMATIQUES (Trigonométrie)} 
\end{itemize}

\section{Mots-clés}

\begin{table}[h]
    \centering
    \label{tab:keywords}
    \begin{tabular}{l|l}
    Français & English \\\hline
    Fracture osseuse & Bone fracture \\
    Détection des bords & Edge detection \\
    Apprentissage automatique & Machine learning \\
    Réseau neuronal convolutif & Convolutional neural network  \\
    Vectorisation & Vectorization \\
    \end{tabular}
\end{table}

\section{Bibliographie commentée}

\setstretch{2}
\paragraph{}

La détection automatisée de maladies et blessures, et, plus particulièrement, de fractures osseuses, est devenue
envisageable ces derniers temps. En effet, l'université américaine de Stanford a organisé en 2018 une compétition ---
<<MURA>> --- visant à encourager la création d'algorithmes détectant des fractures.  Ils ont mis à disposition une large
base de données d'exemples pour permettre aux réseaux neuronaux programmés par les candidats de s'entraîner.  Les
résultats de cette compétition montre qu'un réseau neuronal peut dépasser les capacités d'un être humain avec une dizaine
d'années d'expérience\cite{mura-competition}.

Une année plutôt, William Gale et. al. mettent à profit le même principe de réseaux neuronaux afin de détecter des
fractures du bassin\cite{detect-hip}.  Ils utilisent trois réseaux neuronaux convolutifs (\emph{CNN}) consécutifs
pour réduire le nombre d'images et la taille de celles-ci, en ne sélectionnant que les régions où une fracture est
possible, puis un dernier réseau pour effectuer la détection. Les résultats montrent une amélioration de la précision du
réseau par rapport à des essais antérieurs

\paragraph{}

Les deux examples d'utilisation d'apprentissage automatique dans le cadre de la reconaissance de fractures mentionnés
précédemment partagent un point commun: le nombre d'images radiographiques nécéssaire à l'entraînement de tels réseaux
est assez conséquent: MURA\cite{mura-paper} est un set de 40~561 images tirées de 14~863 études, et le réseau de William
Gale et. al. s'est entraîné sur 53~278 images -- l'entièreté des images de bassins capturées par l'Hôpital \emph{Royal
Adelaide} entre 2005 et 2015\cite{detect-hip}.


Chacune de ces images, pour pouvoir être utile à l'apprentissage, doit être étiquettée au préalable: << y'a-t-il
présence de fracture dans cette image ? >>. De plus, la question de la diversité et de l'inclusivité du set de données
collecté (<< Représente-t-il correctement et justement la population globale ? >>) peut s'avérer
complexe\cite{dataset-bias}.

De plus, ces réseaux ne peuvent répondre avec fiabilité qu'à la question de \emph{présence} d'une fracture, la question
d'une \emph{classification} quelconque demanderait \uwave{donc} une taille de set d'entraînement bien plus grande et
difficilement atteignable.

De surcroît, il existe des centaines de classifications différentes -- certaines avec un champ d'application assez
réduit.  Par exemple, 

\begin{itemize}
    \item la classification Garden\cite{garden-paper} ne concerne que les fractures du bassin; 
    \item la classification SALTER\cite{salter} catégorise les fractures chez l'enfant en fonction de l'atteinte au 
        cartilage de croissance, ce qui a un effet conséquent: dans le type le plus grave, c'est-à-dire 
        l'effacement du cartilage par écrasement, le patient pourrait avoir une différence de longueur 
        entre ses deux jambes.
\end{itemize}

\paragraph{}

Mais l'apprentissage automatique n'est pas la seule potentielle solution à ce problème: des techniques d'analyse d'image
plus classiques pourraient fonctionner.  Il s'agira alors~--~du moins dans un premier temps~--~d'identifier et
d'analyser le (ou les) \emph{trait(s) de fracture}.

La détection de motifs géométriques dans une image en noir et blanc est un problème étudié depuis assez longtemps.  Un
algorithme populaire est celui de la transformée de Hough, d'abord brevetée en 1962 par Paul
Hough\cite{hough-transform-patent} puis généralisée à la détection de lignes et courbes en 1972 par Richard Duda et
Peter Hart\cite{hough-transform}.

Cependant, la détection de lignes dans une image est plus simple quand l'image est bien contrastée, et ne présente que
les bords, et non -- par exemple -- des << lignes >> dues à la porosité du tissu osseux sur des images radiographiques.
En 1986, John F. Canny met au point un algorithme de détection des bords\cite{canny}, qui permet de passer d'une image
source à une image noir et blanc, ou les pixels blancs font partie de bords et les pixels noirs non.




\wordcount{650}

\section{Problématique retenue}

% Je me propose d'étudier la détection de fractures à l'aide de moyens traditionnels, puis avec de l'apprentissage
%automatique. 

Il s'agit d'étudier différents moyens de détection de fractures, et en particulier de comparer une approche par
traitements d'images classiques à une approche avec réseaux neuronaux.


\wordcount{50}

\setstretch{1.5}
\section{Objectifs du TIPE}
Je me propose:
\begin{itemize}
    \item D'explorer la détection et classification sans apprentissage automatique, mais avec des techniques d'analyse
        d'image plus classiques (détection de bords, identification de segments \uwave{de droites}, calcul d'angles),
    \item D'utiliser un ou plusieurs réseaux neuronaux afin d'effectuer la même tâche, et
    \item De comparer les deux approches, en mettant en valeurs les avantages et inconvénients de chacune. 
\end{itemize}

\wordcount{100}

\setstretch{1}
\printbibliography
\end{document}
