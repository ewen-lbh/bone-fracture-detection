\documentclass[a4paper]{article}
\usepackage[a4paper, total={6.5in, 9.5in}]{geometry}
\usepackage[utf8]{inputenc}
\usepackage[T1]{fontenc}
\usepackage[french]{babel}
\usepackage[bookmarks, hidelinks, unicode]{hyperref}
\usepackage{amsmath}
\usepackage{xfrac}
\usepackage{biblatex}
\usepackage{csquotes}
\addbibresource{../sources.bib}
\title{Détection et classification de fractures osseuses}
\author{Ewen Le Bihan}
\date{2021-12-20}

\newcommand\wordcount[1]{
    \immediate\write18{texcount -sub=section \jobname.tex  | grep "Section" | sed -e 's/+.*//' | sed -n \thesection p > 'count.txt'}
($\sfrac{\input{count.txt}}{#1}$ mots)}

\begin{document}
\maketitle

\noindent
\begin{description}
    \item[Professeurs encadrants du candidat]  \hfill
        \begin{itemize}
            \item M. Chireux
            \item M. Lauront
        \end{itemize}
    \item[Liste des membres] \hfill
        \begin{itemize}
            \item LAROCHE Jérémy
            \item LE BIHAN Ewen
        \end{itemize}
\end{description}

\section{Ancrage au thème}



\wordcount{50}

\section{Motivation du choix}

\wordcount{50}


\section{Positionnements thématiques}

\begin{itemize}
    \item \emph{INFORMATIQUE (Traitement d'images)}
    \item \emph{INFORMATIQUE (Aprentissage automatique)}
    \item \emph{MATHÉMATIQUES (Trigonométrie)} 
\end{itemize}

\section{Mots-clés}

\begin{table}[h]
    \centering
    \label{tab:keywords}
    \begin{tabular}{l|l}
    Français & English \\\hline
    Fracture osseuse & Bone fracture \\
    Détection des bords & Edge detection \\
    Vectorisation & Vectorisation \\
    Réseau neuronal & Neural network \\
    Réseau neuronal convolutif & Convolutional neural network 
    \end{tabular}
\end{table}

\section{Bibliographie commentée}

La détection automatisée de fractures osseuses et --- plus généralement --- de maladies est devenue 
envisageable ces derniers temps. L'université américaine de Stanford a organisé en 2018 une compétition visant à
encourager la création d'algorithmes de détections de fractures à partir d'images de scanners\cite{mura-competition}.
Ils ont mis à disposition une large base de données d'exemples pour permettre aux réseaux neuronaux programmés par
les candidats de s'entraîner\cite{mura-paper}.





\wordcount{650}

\section{Problématique retenue}

\wordcount{50}

\section{Objectifs du TIPE}

\wordcount{100}

\section{Liste de références bibliographiques}

\printbibliography
\end{document}
