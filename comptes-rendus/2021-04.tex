\documentclass{article}
\usepackage[utf8]{inputenc}
\usepackage[bookmarks]{hyperref}
\usepackage[margin=0.7in, a4paper]{geometry}
\usepackage{amsmath, amssymb}
\usepackage[french]{babel}


\title{Assistance automatisée au contrôle d'imagerie médicale}
\author{Ewen Le Bihan \textless hey@ewen.works\textgreater}
\date{24 Avril 2021}

\begin{document}
\maketitle

\section{Motivations pour le choix du sujet}

L'intelligence artificielle, et plus précisémment le \emph{machine learning} sont des champs de recherche
qui ont beaucoup de nouvelles avancées ces dernières années.

Les applications à la médecine de ces nouvelles méthodes d'approche de problèmes sont nombreuses.

Il me semblait intéréssant d'aborder ce sujet, et de comparer des approches plus traditionnelles à des approches
incorporant des réseaux neuronaux pour juger de leur efficacité.

\section{Ancrage au thème de l'année}

Le thème de cette année -- ``Santé et prévention'' -- englobe les thèmes médicaux, ce qui ancre bien notre sujet
au thème. 

\section{Bibliographie}

\begin{itemize}
	\item \url{https://neptune.ai/blog/image-processing-in-python-algorithms-tools-and-methods-you-should-know}
\end{itemize}

\section{Déroulé opérationnel}

\emph{Du plus ancien au plus récent} 

\subsection{Travail effectué}


\begin{enumerate}
\item Installation d'un bibliothèque pour lire des fichiers DICOM: problèmes de compilation avec \emph{hdf5}, bibliothèque utilisée par \emph{pydicom} 
\item Utilisation d'une application standalone pour convertir les dicom en png: je me rend compte que le seul dataset disponible gratuitement et librement
  n'est pas intéréssant.
\item Recherche d'images de radios sur internet, directement (google images)
\item 
\end{enumerate}

\subsection{Avant la fin de l'année scolaire}

\begin{enumerate}
	\item Mettre au point un programme de reconaissance de fractures à partir d'images radiologiques
\end{enumerate}

\subsection{Pendant l'été}

\begin{enumerate}
	\item 
\end{enumerate}

\subsection{En deuxième année}

\section{Collaboration}

J'effectue mon TIPE en collaboration avec Jérémy Laroche. 

\end{document}
