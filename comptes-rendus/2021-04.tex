\documentclass{article}
\usepackage[utf8]{inputenc}
\usepackage[bookmarks]{hyperref}
\usepackage[margin=0.7in, a4paper]{geometry}
\usepackage{amsmath, amssymb}
\usepackage[french]{babel}

\title{Assistance automatisée au contrôle d'imagerie médicale}
\author{Ewen Le Bihan \textless hey@ewen.works\textgreater}
\date{24 Avril 2021}

\begin{document}
\maketitle

\section{Motivations pour le choix du sujet}

L'intelligence artificielle, et plus précisémment le \emph{machine learning} sont des champs de recherche
qui ont beaucoup de nouvelles avancées ces dernières années.

Les applications à la médecine de ces nouvelles méthodes d'approche de problèmes sont nombreuses.

Il me semblait intéréssant d'aborder ce sujet, et de comparer des approches plus traditionnelles à des approches
incorporant des réseaux neuronaux afin de juger de leur efficacité.

\section{Ancrage au thème de l'année}

Le thème de cette année -- ``Santé et prévention'' -- englobe les thèmes médicaux, ce qui ancre bien notre sujet
au thème. En ce moment plus que jamais, l'attention des médecins est très demandée, ce qui pousse certains à travailler au-delà de leurs limites.

L'assistance au contrôle de radiographies pourrait aider à prendre les bonnes décisions dans des moments critiques.

\section{Bibliographie}

\begin{itemize}
	\item Bases en traitement d'image avec python \\ {\small \url{https://neptune.ai/blog/image-processing-in-python-algorithms-tools-and-methods-you-should-know}} \\[0.1pt]
	\item Documentation de \emph{pydicom} -- librarie Python permettant la manipulation de fichiers DICOM \\ \url{https://pydicom.github.io/pydicom/stable/}\\[0.1pt]
	\item Banque d'images au format DICOM de tomodensitométries d'os \\ \url{http://isbweb.org/data/vsj/}\\[0.1pt]
	\item Images radiologiques d'os fracturés {\footnotesize (\url{https://duckduckgo.com/?q=broken+bone+radio&iar=images})}
		\begin{itemize}
			\item \url{https://images.radiopaedia.org/images/5848770/fd408943f11320ae39d8631a0b0588.png}
			\item \url{https://i2.wp.com/coreem.net/content/uploads/2017/01/Chauffer-Fracture-Radiopaedia-2.jpg}
			\item \url{https://i.pinimg.com/originals/1d/56/81/1d5681852465ecafe357be3cfb3d5ebe.jpg}
		\end{itemize} %\\[0.1pt]
\end{itemize}

\section{Déroulé opérationnel}

\emph{Par ordre chronologique croissant} 

\subsection{Travail effectué}


\begin{enumerate}
\item Installation d'un bibliothèque pour lire des fichiers DICOM: problèmes de compilation avec \emph{hdf5}, bibliothèque utilisée par \emph{pydicom} 
\item Utilisation d'une application standalone afin de convertir les dicom en png: je me rend compte que le seul dataset\footnote{\url{http://isweb.org/data/vsj/}} disponible gratuitement et librement que j'ai trouvé
  n'est pas à priori intéréssant.
\item Recherche d'images de radios sur internet, directement (google images)
\item Lecture d'articles pour apprendre les bases du traitement d'images en Python
\end{enumerate}

\subsection{Avant la fin de l'année scolaire}

\begin{enumerate}
	\item Mettre au point un programme de reconaissance de fractures à partir d'images radiologiques
\end{enumerate}

\subsection{Pendant l'été}

\begin{enumerate}
	\item Se documenter sur les différents types de réseaux neuronaux adaptés au traitement d'images et les comprendre
	\item Implémenter un ou plusieurs réseaux neuronaux effectuant les même tâches que le programme classique
\end{enumerate}

\subsection{En deuxième année}
\begin{enumerate}
	\item Comparer et interpréter les différences en performances entre les deux approches
\end{enumerate}
\section{Collaboration}

J'effectue mon TIPE en collaboration avec Jérémy Laroche. 
Tandis que j'étudie le traitement d'images, il s'intéresse à l'extraction d'informations de signaux.

\end{document}
